\section{Copland Phrase and Evidence Grammars}
\label{sec:json}

\colorlet{punct}{red!60!black}
\definecolor{background}{HTML}{EEEEEE}
\definecolor{delim}{RGB}{20,105,176}
\colorlet{numb}{magenta!60!black}

\lstdefinelanguage{json}{
    basicstyle=\normalfont\ttfamily,
    numbers=none,
    numberstyle=\scriptsize,
    stepnumber=1,
    numbersep=8pt,
    showstringspaces=false,
    breaklines=true,
    frame=lines,
    backgroundcolor=\color{background},
    literate=
     *{0}{{{\color{numb}0}}}{1}
      {1}{{{\color{numb}1}}}{1}
      {2}{{{\color{numb}2}}}{1}
      {3}{{{\color{numb}3}}}{1}
      {4}{{{\color{numb}4}}}{1}
      {5}{{{\color{numb}5}}}{1}
      {6}{{{\color{numb}6}}}{1}
      {7}{{{\color{numb}7}}}{1}
      {8}{{{\color{numb}8}}}{1}
      {9}{{{\color{numb}9}}}{1}
      {:}{{{\color{punct}{:}}}}{1}
      {,}{{{\color{punct}{,}}}}{1}
      {\{}{{{\color{delim}{\{}}}}{1}
      {\}}{{{\color{delim}{\}}}}}{1}
      {[}{{{\color{delim}{[}}}}{1}
      {]}{{{\color{delim}{]}}}}{1},
    }

\begin{figure}
  \[\begin{array}{rcl}
      \terms&\gets&A\mid\at{p}{t}\mid(t \linseqe t)\mid
                    (t\braseqe\pi t)\mid(t\brapare\pi t) \\
      A&\gets&\asptnn{\bar{a}}
               \mid\cpy\mid\sig\mid\hsh %\mid\cdots      
    \end{array}\]
  \caption{Copland Phrase grammar where: \newline
    $\bar{a}$ = ($m$, $\bar{s}$,
    $p$, $r$); $m = \aspid\in\mathbb{N}$; $\bar{s}$ is
    a list of string arguments; $p
    = \plc\in\mathbb{N}$; $r = target\_id\in\mathbb{N}$;  and $\pi = (\pi_1,\pi_2)$ is a pair of
    evidence splitting functions.
}
  \label{fig:term-grammar-json}
\end{figure}


\begin{figure}
  %\begin{scriptsize}
  \[\begin{array}{rcl}
      \evt&\gets&\mtc\mid\Nt{\nidt}\mid\Ut{\bar{a}}{p}{\evt} \\
          & & \mid\Gt{p}{\evt}\mid\Ht{p}{\evt} \\
          & & \mid\SSt{\evt}{\evt}\mid\PPt{\evt}{\evt} %\mid\cdots
    \end{array}\]
  \caption{Evidence Type grammar where: \newline
    $\bar{a}$ and $p$ are as in Fig.~\ref{fig:term-grammar-json} and
    $n = \nid\in\mathbb{N}$.}
  \label{fig:evidence_type-json}
  %\end{scriptsize}
\end{figure}


\begin{figure}
  %\begin{small}
  \[\begin{array}{rcl}
      \evt_{c}&\gets&\mtc_{c}\mid\Ntc{\nidt}{\bs}\mid\Utc{\bar{a}}{p}{\bs}{\evt_{c}}\\
              & & \mid\Gtc{p}{\bs}{\evt_{c}}\mid\Htc{p}{\bs}{\evt} \\
      & & \mid\SSt{\evt_{c}}{\evt_{c}}\mid\PPt{\evt_{c}}{\evt_{c}} %\mid\cdots
      
    \end{array}\]
  \caption{Typed Concrete Evidence grammar where: \newline
     $\bar{a}$, $p$, and $n$ are as in Fig.~\ref{fig:evidence_type-json}
     and $\bs\in\bst$~(binary values).}
  \label{fig:concrete-ev-json}
  %\end{small}
\end{figure}

\newpage

\section{JSON Schemas}
We represent Copland terms ($\terms$ in
Figure~\ref{fig:term-grammar-json}), Evidence Types ($\evt$ in
Figure~\ref{fig:evidence_type-json}), and Typed Concrete Evidence
($\evt_{c}$ in Figure~\ref{fig:concrete-ev-json}) as Algebraic Data
Types (ADTs) in the Attestation Manager prototypes.  In JSON we
represent each ADT as an object with two fields:

\begin{enumerate}
  \item $\cnc{constructor}$-the constructor name as a JSON
    string ($\strj$).  Constructor names must be unique for unambiguous parsing.
  \item $\cnc{data}$-An \textbf{\emph{ordered}} JSON array ($\arrj$) that holds
the arguments for that particular constructor.  Members of the data
array will differ from constructor to constructor.
\end{enumerate}

\noindent The general schema for ADTs (labelled by placeholder
$\adtj$) is as follows: \\

\begin{lstlisting}[language=json, escapeinside={(*}{*)}]
{
  "constructor": (*\strj*),
  "data": (*\arrj*) | (*\adtj*)
}
\end{lstlisting}

\noindent The $\adtj$ alternative for the $\cnc{data}$ field accounts for
degenerate nesting of constructors (for example in the ASP constructors below).

% Constructor names and data arrays for each ADT are specified in
% schemas below.    Note that $\strj$ and $\arrj$ are the JSON primitive
% string and array types.

\subsection{Copland Phrase JSON Schemas}
The following JSON object schemas correspond to the constructors of
the Copland phrase grammar ($\terms$) in
Figure~\ref{fig:term-grammar-json}, and satisfy the $\tj$ placeholder.
The $\numj$ and $\strj$ placeholders are for the standard json number
and string datatypes.  The order of items in the ``data'' subarray is
significant--they match the order of arguments to each constructor.
Finally the placeholder $\spj$ stands for an evidence splitter, a JSON
string with one of the constant values ``ALL'' or ``NONE''. \\

% {"data":{"data":[1,["arg1","arg2"],2,3],"constructor":"ASPC"},"constructor":"Coq_asp"}

\noindent $\aspcj$~:=~$[~\numj,~[\strj],~\numj,~\numj~]$ \\

\begin{lstlisting}[language=json, escapeinside={(*}{*)}]
{
  "constructor": "Coq_asp",
  "data": {"constructor": "ASPC",
           "data": (*\aspcj*)
           }
}
\end{lstlisting}

%\newpage

\begin{lstlisting}[language=json, escapeinside={(*}{*)}]
{
  "constructor": "Coq_asp",
  "data": {"constructor": "CPY"}
}
\end{lstlisting}

\begin{lstlisting}[language=json, escapeinside={(*}{*)}]
{
  "constructor": "Coq_asp",
  "data": {"constructor": "SIG"}
}
\end{lstlisting}

\begin{lstlisting}[language=json, escapeinside={(*}{*)}]
{
  "constructor": "Coq_asp",
  "data": {"constructor": "HSH"}
}
\end{lstlisting}

\begin{lstlisting}[language=json, escapeinside={(*}{*)}]
{
  "constructor": "Coq_att",
  "data": [ (*\numj*),
            (*\tj*) ]
}
\end{lstlisting}
  
\begin{lstlisting}[language=json, escapeinside={(*}{*)}]
{
  "constructor": "Coq_lseq",
  "data": [ (*\tj*),
            (*\tj*) ]
}
\end{lstlisting}

\vspace{2mm}

\noindent $\spj$ := ``ALL''~$\mid$~``NONE'' \\

\begin{lstlisting}[language=json, escapeinside={(*}{*)}]
{
  "constructor": "Coq_bseq",
  "data": [ [(*\spj*), (*\spj*)],
             (*\tj*),
             (*\tj*) ]
}
\end{lstlisting}

\begin{lstlisting}[language=json, escapeinside={(*}{*)}]
{
  "constructor": "Coq_bpar",
  "data": [ [(*\spj*), (*\spj*)],
             (*\tj*),
             (*\tj*) ]
}
\end{lstlisting}

\newpage

\subsection{Copland Evidence Type Schemas}
The following JSON object schemas correspond to the constructors of
the Evidence Type grammar ($\evt$) in Figure~\ref{fig:evidence_type-json}, and satisfy the $\ej$ placeholder.

%\newpage

\begin{lstlisting}[language=json, escapeinside={(*}{*)}]
{
  "constructor": "Coq_mt"
}
\end{lstlisting}

\begin{lstlisting}[language=json, escapeinside={(*}{*)}]
{
  "constructor": "Coq_nn",
  "data": [ (*\numj*) ]
}
\end{lstlisting}

\begin{lstlisting}[language=json, escapeinside={(*}{*)}]
{
  "constructor": "Coq_uu",
  "data": [ (*\aspcj*),
            (*\numj*),
            (*\ej*) ]
}
\end{lstlisting}

\begin{lstlisting}[language=json, escapeinside={(*}{*)}]
{
  "constructor": "Coq_gg",
  "data": [ (*\numj*),
            (*\ej*) ]
}
\end{lstlisting}

\begin{lstlisting}[language=json, escapeinside={(*}{*)}]
{
  "constructor": "Coq_hh",
  "data": [ (*\numj*),
            (*\ej*) ]
}
\end{lstlisting}

\begin{lstlisting}[language=json, escapeinside={(*}{*)}]
{
  "constructor": "Coq_ss",
  "data": [ (*\ej*),
            (*\ej*) ]
}
\end{lstlisting}

\begin{lstlisting}[language=json, escapeinside={(*}{*)}]
{
  "constructor": "Coq_pp",
  "data": [ (*\ej*),
            (*\ej*) ]
}
\end{lstlisting}


\subsection{Copland Typed Concrete Evidence Schemas}
The following JSON object schemas correspond to the constructors of
the Typed Concrete Evidence grammar ($\evt_{c}$) in
Figure~\ref{fig:concrete-ev-json}, and satisfy the $\ejc$ placeholder.
Note: Constructor fields that hold binary data ($\cnc{bs}$ parameters
in the grammar) become base64-encoded JSON strings ($\bsj$), holding arbirary
binary data--hashes, nonces, signatures, etc: \\

\noindent $\bsj$ := $\strj$ (Base64 encoded) \\

\begin{lstlisting}[language=json, escapeinside={(*}{*)}]
{
  "constructor": "Coq_mtc"
}
\end{lstlisting}

\begin{lstlisting}[language=json, escapeinside={(*}{*)}]
{
  "constructor": "Coq_nnc",
  "data": [ (*\numj*), (*\bsj*) ]
}
\end{lstlisting}

\begin{lstlisting}[language=json, escapeinside={(*}{*)}]
{
  "constructor": "Coq_uuc",
  "data": [ (*\aspcj*),
            (*\numj*),
            (*\bsj*),
            (*\ejc*) ]
}
\end{lstlisting}

\begin{lstlisting}[language=json, escapeinside={(*}{*)}]
{
  "constructor": "Coq_ggc",
  "data": [ (*\numj*),
            (*\bsj*)
            (*\ejc*) ]
}
\end{lstlisting}

\begin{lstlisting}[language=json, escapeinside={(*}{*)}]
{
  "constructor": "Coq_hhc",
  "data": [ (*\numj*),
            (*\bsj*)
            (*\ej*) ]
}
\end{lstlisting}

\newpage

\begin{lstlisting}[language=json, escapeinside={(*}{*)}]
{
  "constructor": "Coq_ssc",
  "data": [ (*\ejc*),
            (*\ejc*) ]
}
\end{lstlisting}

\begin{lstlisting}[language=json, escapeinside={(*}{*)}]
{
  "constructor": "Coq_ppc",
  "data": [ (*\ejc*),
            (*\ejc*) ]
}
\end{lstlisting}


\newcommand{\mapadd}{\ensuremath{p => Address}}

\subsection{Message Schemas}


\begin{figure}

\[\begin{array}{rcl}
      RequestMessage = \{\\
      toPlace &:: &\cnc{p},\\
      fromPlace &:: &\cnc{p},\\
      reqNameMap &:: &\cnc{\mapadd},\\
      reqTerm &:: &\cnc{t},\\
      reqEv &:: &[~\bs~]~\}
  \end{array}\]

\[\begin{array}{rcl}
      ResponseMessage = \{\\
      respToPlace &:: &\cnc{p},\\
      respFromPlace &:: &\cnc{p},\\
      respEv &:: &[~\bs~]~\}
  \end{array}\]

  \caption{Request and Response Message record structures.}\label{fig:messages-grammar}
\end{figure}

\colorlet{punct}{red!60!black}
\definecolor{background}{HTML}{EEEEEE}
\definecolor{delim}{RGB}{20,105,176}
\colorlet{numb}{magenta!60!black}

\lstdefinelanguage{json}{
    basicstyle=\normalfont\ttfamily,
    numbers=none,
    numberstyle=\scriptsize,
    stepnumber=1,
    numbersep=8pt,
    showstringspaces=false,
    breaklines=true,
    frame=lines,
    backgroundcolor=\color{background},
    literate=
     *{0}{{{\color{numb}0}}}{1}
      {1}{{{\color{numb}1}}}{1}
      {2}{{{\color{numb}2}}}{1}
      {3}{{{\color{numb}3}}}{1}
      {4}{{{\color{numb}4}}}{1}
      {5}{{{\color{numb}5}}}{1}
      {6}{{{\color{numb}6}}}{1}
      {7}{{{\color{numb}7}}}{1}
      {8}{{{\color{numb}8}}}{1}
      {9}{{{\color{numb}9}}}{1}
      {:}{{{\color{punct}{:}}}}{1}
      {,}{{{\color{punct}{,}}}}{1}
      {\{}{{{\color{delim}{\{}}}}{1}
      {\}}{{{\color{delim}{\}}}}}{1}
      {[}{{{\color{delim}{[}}}}{1}
      {]}{{{\color{delim}{]}}}}{1},
    }

%\section{Data Exchange Format (JSON Schema)}

%\subsection{Message Schemas}
We represent Request and Response Messages as record structures in Figure~\ref{fig:messages-grammar}.  Their respective JSON object schemas are as follows: \\


\begin{lstlisting}[language=json, escapeinside={(*}{*)}]
{
  "toPlace": (*\numj*),
  "fromPlace": (*\numj*),
  "reqNameMap": (*\namej*),
  "reqTerm": (*\tj*),
  "reqEv": (*\rawevj*)
}
\end{lstlisting}

\newpage

\begin{lstlisting}[language=json, escapeinside={(*}{*)}]
{
  "respToPlace": (*\numj*),
  "respFromPlace": (*\numj*),
  "respEv": (*\rawevj*)
}
\end{lstlisting}

% \noindent $\numj$ is the JSON primitive number type.  $\tj$ and
% $\ej$ are JSON objects for Copland terms and evidence
% respectively. \\

\vspace{2mm}

\noindent where \\ \\
\noindent $\rawevj$ := $[~\bsj~]$ \\

\noindent $\namej$ is a JSON object of the following form:

\{$pl_1$:$addr_1$, $pl_2$:$addr_2$, ...\} \\

\noindent where $pl_1$, $pl_2$, ... are JSON key strings that
represent a Copland place identifier (i.e. ``1'', ``2'', ...) and
$addr_1$, $addr_2$, ... are JSON strings ($\strj$) that represent
platform addresses.  We leave address strings abstract in this
specification, but an example usage would be a string of the form
ip:port.

We represent Sig Request and Response Messages as record structures in
Figure~\ref{fig:sig-messages-grammar}.  Their respective JSON object
schemas are as follows: \\ \\

%\newpage

\begin{lstlisting}[language=json, escapeinside={(*}{*)}]
{
  "evBits": (*\bsj*)
}
\end{lstlisting}

\begin{lstlisting}[language=json, escapeinside={(*}{*)}]
{
  "sigBits": (*\bsj*)
}
\end{lstlisting}

\begin{figure}

\[\begin{array}{rcl}
      SigRequestMessage = \{ evBits &:: &\bs \}
  \end{array}\]

\[\begin{array}{rcl}
      SigResponseMessage = \{ sigBits &:: &\bs \}
  \end{array}\]

  \caption{Sig Request and Response Message record structures.}\label{fig:sig-messages-grammar}
\end{figure}

% \vspace{4mm}
\newpage

We represent Asp Request and Response Messages as record structures in
Figure~\ref{fig:asp-messages-grammar}.  Their respective JSON object
schemas are as follows: \\ \\

\begin{lstlisting}[language=json, escapeinside={(*}{*)}]
{
  "aspArgs": (*\aspcj*),
  "aspInputEv": (*\rawevj*)
}
\end{lstlisting}

\begin{lstlisting}[language=json, escapeinside={(*}{*)}]
{
  "aspBits": (*\bsj*)
}
\end{lstlisting}

\begin{figure}

\[\begin{array}{rcl}
      AspRequestMessage = \{\\
      aspArgs &:: &\cnc{\bar{a}},\\
      aspInputEv &:: &[~\bs~]~\}
  \end{array}\]

\[\begin{array}{rcl}
      AspResponseMessage = \{ aspBits &:: &\bs \}
  \end{array}\]

  \caption{Asp Request and Response Message record structures.}\label{fig:asp-messages-grammar}
\end{figure}




%%% Local Variables: ***
%%% mode:latex ***
%%% TeX-master: "copland_json.tex"  ***
%%% End: ***
